%!TEX root = main.tex

Der Datensatz \emph{adult} enth\"alt sehr viele Attribute, die stark miteinander korrelieren, wie zum Beispiel \emph{sex} und \emph{marital-status}. Daher haben wir, um spannende Regeln zu finden sehr viele Attribute entfernt, und listen nur die Attribute auf, die wir zum Finden von Regeln verwendet haben.

\vspace{0.1cm} 

\begin{boxedminipage}{\textwidth}
\begin{tabular}{rl}
\emph{Attribute:}  & age, workclass, education, occupation, relationship, \\
                   & race, hoursperweek, native-country, class \\
\emph{Sortierung:} &  lift \\
\end{tabular}
\vspace{0.3cm}

$\textbf{relationship} = \textbf{Not-in-family} \Longrightarrow \textbf{class} = \textbf{<=50K} $ (confidence: 0.9 lift: 1.18)
\end{boxedminipage} \vspace{0.1cm}

\vspace{0.1cm} 

\begin{boxedminipage}{\textwidth}
\begin{tabular}{rl}
\emph{Attribute:}  & workclass, education, occupation, race, hoursperweek, class \\
\emph{Sortierung:} &  confidence \\
\end{tabular}
\vspace{0.3cm}
$\textbf{class} = \textbf{>50K} \Longrightarrow \textbf{race} = \textbf{White}$    (confidence: 0.91)
\end{boxedminipage} \vspace{0.1cm}

Da etwa 85 \% der Teilnehmer der Studie \textbf{race} = \textbf{white} hatten, wurden sehr viele Regeln der Form $\Longrightarrow$ \textbf{race} = \textbf{white} gefunden. Entfernt man dieses Attribut, findet man noch weitere interessante Regeln.

\vspace{0.1cm} 

\begin{boxedminipage}{\textwidth}
\begin{tabular}{rl}
\emph{Attribute:}  & age, education, marital-status, occupation \\
\emph{Sortierung:} &  confidence \\
\end{tabular}
\vspace{0.3cm}

$\textbf{age}= \mathbf{0} \Longrightarrow \textbf{marital-status}= \textbf{Never-married}$    (confidence: 0.85)

$\textbf{age}= \mathbf{3} \Longrightarrow \textbf{marital-status}=\textbf{Married-civ-spouse}$  (  confidence: 0.62)
\end{boxedminipage} \vspace{0.1cm}


Wobei hier \textbf{age} wahrscheinlich nicht f\"ur das tas\"achliche alter steht, sondern f\"ur bestimmte Altersgrenzen. Junge Menschen sind nicht oft verheiratet. Alte M\"anner sind oft verheiratet.

\vspace{0.1cm}

\begin{boxedminipage}{\textwidth}
\begin{tabular}{rl}
\emph{Attribute:}  & marital-status, relationship, class \\
\emph{Sortierung:} &  confidence \\
\end{tabular}
\vspace{0.3cm}

$\textbf{relationship}=\textbf{Own-child} \Longrightarrow \textbf{class} = \textbf{<=50K} $   (confidence: 0.99)

$\textbf{class}=\textbf{>50K} \Longrightarrow \textbf{marital-status}=\textbf{Married-civ-spouse}$ (confidence: 0.85)
\end{boxedminipage} \vspace{0.1cm}

Erwachsene die ein eigenes Kind haben, verdienen eher wenig. Erwachsene die viel verdienen sind oft verheiratete M\"anner.





F\"ur diese Aufgabe benutzen wir die Datens\"atze \emph{glass, iris} und \emph{splice}.
\emph{glass} wurde f\"ur den Prism-Learner mit dem Discretize-Filter verwendet (\emph{java weka.filters.supervised.attribute.Discretize -i glass.arff -o glass\_nom.arff -R 1,2,3,4,5,6,7,8,9 -c last}).\\

Anzahl der Regeln \\
\begin{tabular}{c|c|c|c}
	             & glass & iris & splice \\ \hline
Conjunctive Rule &   1   &  1   &   1    \\ \hline
JRip             &   8   &  4   &   14   \\ \hline
Prism	         &   63  &  16   &  3176    \\
\end{tabular}\\ \\

Gesamtanzahl der Bedingungen \\
\begin{tabular}{c|c|c|c}
	             & glass & iris & splice \\ \hline
Conjunctive Rule &   2   &  1   &   1    \\ \hline
JRip             &  18   &  3   &   55   \\ \hline
Prism	         &  385  & 51   &   3176   \\
\end{tabular}\\ \\

Anzahl der vorhergesagten Klassen \\
\begin{tabular}{c|c|c|c}
	             & glass & iris & splice \\ \hline
Conjunctive Rule &   1   &  1   &   1    \\ \hline
JRip             &   6   &  3   &   3    \\ \hline
Prism	         &   6   &  3   &   3    \\
\end{tabular}\\ \\


Eine Default Rule existiert nur bei JRip. Dort wird als Defaultklasse \"ublicherweise die Klasse gew\"ahlt, die am h\"aufigsten im Datensatz vorkommt. 

Der Datensatz \emph{iris} l\"asst sich am einfachsten lernen, da man hier besonders wenig Regeln und besonders wenig Bedingungen ben\"otigt.

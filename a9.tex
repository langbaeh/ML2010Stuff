%!TEX root = main.tex
Datens\"atze: ionosphere, iris, yeast, letter.

\textbf{Accuracy}

\begin{table}[htb]
	\centering
\begin{tabular}{l|c|c|c|c}
	               & ionosphere & iris  & yeast & letter \\ \hline
J48 unfiltered     &  91.5      &  96.0 &  56.0 &  88.0  \\ \hline
J48 filtered       &  89.2      &  94.0 &  59.1 &  78.6  \\ \hline
FilteredClassifier &  91.2      &  93.3 &  57.0 &  78.7 
\end{tabular}
\end{table}
\textbf{Number of Leaves}

\begin{table}[htb]
	\centering
\begin{tabular}{l|c|c|c|c}
	               & ionosphere & iris  & yeast & letter \\ \hline
J48 unfiltered     &  18        &  5    &  185  &  1226  \\ \hline
J48 filtered       &  21        &  3    &  64   &  9624  \\ \hline
FilteredClassifier &  21        &  3    &  64   &  9624 
\end{tabular}
\end{table}

Der \emph{FilteredClassifier} liefert im Durchschnitt eine geringe Accuracy, die aber Vermutlich realistischer ist, da die Informationen f\"ur das Filtering ausschlie\ss lich aus dem Testteil der Cross-Validation gewonnen werden, und nicht aus dem Trainingsteil. Filtert man die Daten vor dem Lernen benutzt man Informationen aus dem gesamten Datensatz. 

Die Gr\" osse der B\"aume h\"angt nicht von der Art des Filterings ab.


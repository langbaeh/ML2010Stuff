
% TODO: Qualitaet einschaetzen

\begin{tabular}{c|c|c|c|c|c}
				Datensatz         & 1x5  & 1x10 & 1x20 & LOO  & Trainingsmenge   \\ \hline
				\emph{glass}      & 67.3 & 61.8 & 60.7 & 61.7 & 85.98   \\ \hline
				\emph{iris}       & 92.0 & 88.0 & 96.0 & 93.3 & 96.0    \\ \hline
				\emph{audiology}  & 67.3 & 66.4 & 69.9 & 69.9 & 76.1    \\ \hline
				\emph{ionosphere} & 89.2 & 92.0 & 90.0 & 89.2 & 100   \\ \hline
				\emph{yeast}      & 56.5 & 57.7 & 57.7 & 59.4 & 67.8   \\ \hline
\end{tabular}


% TODO: fuerht eine geschickte auswahl zu besseren abschaetzungen?

\begin{tabular}{c|c|c|c|c|c}
				Datensatz         & 10x10    \\ \hline
				\emph{glass}      & 67.3     \\ \hline
				\emph{iris}       & 92.0     \\ \hline
				\emph{audiology}  & 67.3     \\ \hline
				\emph{ionosphere} & 89.2     \\ \hline
				\emph{yeast}      & 56.5     \\ \hline
\end{tabular}


\begin{tabular}{c|c}
				Datensatz         & Validierungsmenge    \\ \hline
				\emph{glass}      & 73.1     \\ \hline
				\emph{iris}       & 93.3     \\ \hline
				\emph{audiology}  & 67.3   \\ \hline
				\emph{ionosphere} & 84.6   \\ \hline
				\emph{yeast}      & 56.1 \\ \hline
\end{tabular}

% TODO: wie war die abschaetzung?
% 10x10 Cross-Validation war grob am genausten
\input{header}


\areaset[66pt]{415pt}{600pt}


\begin{document}
\title{WEKA BLAAAA}
\author{Fabian Langguth, Sebastian Koch}
\date{November 2010}

\maketitle

\section{Aufgabe 1 Regellernen} % (fold)
\label{sec:aufgabe_1_regellernen}

F\"ur diese Aufgabe benutzen wir die Datens\"atze \emph{glass, iris} und \emph{splice}.
\emph{glass} wurde f\"ur den Prism-Learner mit dem Discretize-Filter verwendet (\emph{java weka.filters.supervised.attribute.Discretize -i glass.arff -o glass\_nom.arff -R 1,2,3,4,5,6,7,8,9 -c last}).\\

Anzahl der Regeln \\
\begin{tabular}{c|c|c|c}
	             & glass & iris & splice \\ \hline
Conjunctive Rule &   1   &  1   &   1    \\ \hline
JRip             &   8   &  4   &   14   \\ \hline
Prism	         &   63  &  16   &  3176    \\
\end{tabular}\\ \\

Gesamtanzahl der Bedingungen \\
\begin{tabular}{c|c|c|c}
	             & glass & iris & splice \\ \hline
Conjunctive Rule &   2   &  1   &   1    \\ \hline
JRip             &  18   &  3   &   55   \\ \hline
Prism	         &  385  & 51   &   3176   \\
\end{tabular}\\ \\

Anzahl der vorhergesagten Klassen \\
\begin{tabular}{c|c|c|c}
	             & glass & iris & splice \\ \hline
Conjunctive Rule &   1   &  1   &   1    \\ \hline
JRip             &   6   &  3   &   3    \\ \hline
Prism	         &   6   &  3   &   3    \\
\end{tabular}\\ \\


Eine Default Rule existiert nur bei JRip. Dort wird als Defaultklasse \"ublicherweise die Klasse gew\"ahlt, die am h\"aufigsten im Datensatz vorkommt. 

Der Datensatz \emph{iris} l\"asst sich am einfachsten lernen, da man hier besonders wenig Regeln und besonders wenig Bedingungen ben\"otigt.


% section aufgabe_1_regellernen (end)

\section{Aufgabe 2 Evaluation von Regellernern} % (fold)
\label{sec:aufgabe_2_evaluation_von_regellernern}

\begin{tabular}{c|c|c|c|c|c}
				Datensatz         & 1x5  & 1x10 & 1x20 & LOO  & Trainingsmenge   \\ \hline
				\emph{glass}      & 67.3 & 61.8 & 60.7 & 61.7 & 85.98   \\ \hline
				\emph{iris}       & 92.0 & 88.0 & 96.0 & 93.3 & 96.0    \\ \hline
				\emph{audiology}  & 67.3 & 66.4 & 69.9 & 69.9 & 76.1    \\ \hline
				\emph{ionosphere} & 89.2 & 92.0 & 90.0 & 89.2 & 100   \\ \hline
				\emph{yeast}      & 56.5 & 57.7 & 57.7 & 59.4 & 67.8   \\ \hline
\end{tabular}

\begin{tabular}{c|c|c|c|c|c}
				Datensatz         & 10x10    \\ \hline
				\emph{glass}      & 67.3     \\ \hline
				\emph{iris}       & 92.0     \\ \hline
				\emph{audiology}  & 67.3     \\ \hline
				\emph{ionosphere} & 89.2     \\ \hline
				\emph{yeast}      & 56.5     \\ \hline
\end{tabular}


% section aufgabe_2_evaluation_von_regellernern (end)


\end{document}
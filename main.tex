%!TEX root = main.tex
\documentclass
[
	a4paper,
	11pt,				% 10pt, 11pt oder 12pt - Standard ist 10pt
	pointlessnumbers,	% kein abschlie�enden Punkt hinter den Nummerierungen machen
	%pdftex,
	%chapterprefix,		% Kapitel anschreiben als Kapitel
	%openright, 		
	%openany,
	twoside,
	abstracton,
	final,				%draft,
	%normalheadings,	% �berschriften etwas kleiner (smallheadings)
	%BCOR21mm,			% Bindekorrektur, bspw. 1 cm
	%DIV13,			% f�hrt die Satzspiegelberechnung neu aus
	bibtotoc			% Literaturverzeichniss in toc eintragen
	%bibgerm,			% wegen BibTex in Deutschen Stil, sollte an alle weitergegeben werden
]
{scrreprt}				% KOMA-Script 

%\pagestyle{headings}			% sieht gut aus

% Packages:
%\usepackage[ngerman]{babel}
\usepackage[USenglish]{babel}
\usepackage[T1]{fontenc}		% f�r T1 Zeichensatz
\usepackage{lmodern} 			% Type1-Schriftart f�r nicht-englische Texte

\usepackage{amsmath}
\usepackage{amsthm}
\usepackage{amssymb}
%\usepackage{dsfont}				% brauche ich bis jetzt nur f�r die Zahlenbereiche: N,R,Z 
%\usepackage[babel,german=quotes]{csquotes}	% f�r einheitliche Anf�hrungszeichen

\usepackage{tikz}		% syntax layer for pgf, a tae macro package for generating graphics
\usetikzlibrary{plotmarks}
\usepackage{verbatim}

%\usetikzlibrary{chains,decorations}
\usepackage{pgf}
\usepackage{xcolor}				% f�r Farbmischungen
\usepackage{multicol}			% f�r mehrere Spalten
\usepackage{url}	
\usepackage{nameref}			% um Kapitel�berschriften referenzieren zu k�nnen

%\renewcaptionname{ngerman}{\abstractname}{Kurzfassung}

%\usepackage{caption}	% ich mache lieber alles �ber das Koma Script:
	\setcapindent{0em}	% kein Einzug
	\addtokomafont{caption}{\footnotesize} %\footnotesize \small 
	\addtokomafont{captionlabel}{\sffamily\bfseries}

%% Packages f�r Grafiken & Abbildungen %%%%%%%%%%%%%%%%%%%%%%
\usepackage{graphicx} %%Zum Laden von Grafiken
\usepackage{subfig} %%Teilabbildungen in einer Abbildung
\usepackage{float}		% f�r den [H] exakt hier Specifier
\usepackage{pdfpages}



% wichtig: hyperef muss als letztes Paket geladen werden, weiter Optionen: [pdfstartview={Fit}, pagebackref]
%\xdefinecolor{urlFarbe}{rgb}{0.0,0.5,0.5}
\usepackage[pdfauthor={Fabian Langguth}, bookmarks=false, pdftex, colorlinks=true, urlcolor=urlFarbe, linkcolor=black, citecolor=black]{hyperref}



\areaset[66pt]{415pt}{600pt}


\begin{document}
\title{WEKA BLAAAA}
\author{Fabian Langguth, Sebastian Koch}
\date{November 2010}

\maketitle

\section{Aufgabe 1 Regellernen} % (fold)
\label{sec:aufgabe_1_regellernen}

F\"ur diese Aufgabe benutzen wir die Datens\"atze \emph{glass, iris} und \emph{splice}.
\emph{glass} wurde f\"ur den Prism-Learner mit dem Discretize-Filter verwendet (\emph{java weka.filters.supervised.attribute.Discretize -i glass.arff -o glass\_nom.arff -R 1,2,3,4,5,6,7,8,9 -c last}).\\

Anzahl der Regeln \\
\begin{tabular}{c|c|c|c}
	             & glass & iris & splice \\ \hline
Conjunctive Rule &   1   &  1   &   1    \\ \hline
JRip             &   8   &  4   &   14   \\ \hline
Prism	         &   63  &  16   &  3176    \\
\end{tabular}\\ \\

Gesamtanzahl der Bedingungen \\
\begin{tabular}{c|c|c|c}
	             & glass & iris & splice \\ \hline
Conjunctive Rule &   2   &  1   &   1    \\ \hline
JRip             &  18   &  3   &   55   \\ \hline
Prism	         &  385  & 51   &   3176   \\
\end{tabular}\\ \\

Anzahl der vorhergesagten Klassen \\
\begin{tabular}{c|c|c|c}
	             & glass & iris & splice \\ \hline
Conjunctive Rule &   1   &  1   &   1    \\ \hline
JRip             &   6   &  3   &   3    \\ \hline
Prism	         &   6   &  3   &   3    \\
\end{tabular}\\ \\


Eine Default Rule existiert nur bei JRip. Dort wird als Defaultklasse \"ublicherweise die Klasse gew\"ahlt, die am h\"aufigsten im Datensatz vorkommt. 

Der Datensatz \emph{iris} l\"asst sich am einfachsten lernen, da man hier besonders wenig Regeln und besonders wenig Bedingungen ben\"otigt.


% section aufgabe_1_regellernen (end)

\section{Aufgabe 2 Evaluation von Regellernern} % (fold)
\label{sec:aufgabe_2_evaluation_von_regellernern}

\begin{tabular}{c|c|c|c|c|c}
				Datensatz         & 1x5  & 1x10 & 1x20 & LOO  & Trainingsmenge   \\ \hline
				\emph{glass}      & 67.3 & 61.8 & 60.7 & 61.7 & 85.98   \\ \hline
				\emph{iris}       & 92.0 & 88.0 & 96.0 & 93.3 & 96.0    \\ \hline
				\emph{audiology}  & 67.3 & 66.4 & 69.9 & 69.9 & 76.1    \\ \hline
				\emph{ionosphere} & 89.2 & 92.0 & 90.0 & 89.2 & 100   \\ \hline
				\emph{yeast}      & 56.5 & 57.7 & 57.7 & 59.4 & 67.8   \\ \hline
\end{tabular}
% TODO: Qualitaet einschaetzen

\begin{tabular}{c|c|c|c|c|c}
				Datensatz         & 10x10    \\ \hline
				\emph{glass}      & 67.3     \\ \hline
				\emph{iris}       & 92.0     \\ \hline
				\emph{audiology}  & 67.3     \\ \hline
				\emph{ionosphere} & 89.2     \\ \hline
				\emph{yeast}      & 56.5     \\ \hline
\end{tabular}

% TODO: fuerht eine geschickte auswahl zu besseren abschaetzungen?

\begin{tabular}{c|c}
				Datensatz         & Validierungsmenge    \\ \hline
				\emph{glass}      & 73.1     \\ \hline
				\emph{iris}       & 93.3     \\ \hline
				\emph{audiology}  & 67.3   \\ \hline
				\emph{ionosphere} & 84.6   \\ \hline
				\emph{yeast}      & 56.1 \\ \hline
\end{tabular}

% TODO: wie war die abschaetzung?
% 10x10 Cross-Validation war grob am genausten

% section aufgabe_2_evaluation_von_regellernern (end)
\section{Aufgabe 3 ROC-Kurven}
\label{sec:aufgabe_3_ROC_kurven}

Datensatz: glass

\begin{tabular}{c|c|c|c|c}
				Regellerner       & \emph{build wind float} & \emph{containers} & \emph{tableware}  \\ \hline
				\emph{J48}			& 0.81 & 0.87 & 0.93  \\ \hline
				\emph{Naive Bayes}  & 0.71 & 0.84 & 0.98  
\end{tabular}


% section aufgabe_3_ROC_kurven (end)
\section{Aufgabe 4 Entscheidungsb\"aume}
\label{sec:aufgabe_4_entscheidungsbaume}

Datensatz: contact lenses, kr-vs-kp

Area under ROC curve

\begin{tabular}{c|c|c|c}
				Regellerner       & \emph{soft} & \emph{hard} & \emph{none}  \\ \hline
				\emph{J48 - unpruned} &  &  &   \\ \hline
				\emph{J48 - pruned}  &  &  &  \\ \hline
				\emph{ID3}  &  &  &  \\ \hline
\end{tabular}

Accuracy

\begin{tabular}{c|c|c}
				Regellerner       & \emph{contact lenses} & \emph{kr-vs-kp}  \\ \hline
				\emph{J48 - pruned} &   &   \\ \hline
				\emph{J48 - unpruned}  & &  \\ \hline
				\emph{ID3}  &  \\ \hline
\end{tabular}




%section aufgabe_4_entscheidungsbaume (end)

\section{Aufgabe 5 Nearest Neighbour}
\label{sec:aufgabe_4_nearest_neighbour}

Accuracy

\begin{tabular}{c|c|c}
                k-NN       & \emph{contact lenses} & \emph{kr-vs-kp}  \\ \hline
				\emph{k = 1} & 79.17  & 96.28  \\ \hline
				\emph{k = 3} & 79.17  & 96.50  \\ \hline
				\emph{k = 5} & 66.67  & 96.03  \\ \hline
				\emph{k = 7} & 58.33  & 95.40  \\ \hline
				\emph{k = 9} & 58.33  & 95.24  \\ \hline
				\emph{k = 11} & 58.33  & 95.06  \\ \hline
\end{tabular}

%section aufgabe_4_nearest_neighbour (end)


\section{Aufgabe 6 Regressionsb\"aume}
\label{sec:aufgabe_6_regressionsbaume}


Commandlines: 

regression tree, unpruned
java -cp WEKAPATH weka.classifiers.trees.M5P -R -N -i -t Regression/housing.arf

regression tree, pruned
java -cp WEKAPATH weka.classifiers.trees.M5P -R -i -t Regression/housing.arf

model tree, unpruned
java -cp WEKAPATH weka.classifiers.trees.M5P -i -N -t Regression/housing.arf

model tree, pruned
java -cp WEKAPATH weka.classifiers.trees.M5P -i -t Regression/housing.arf

\begin{table}
\begin{tabular}{c|c|c|c|c|c|c|c|c}
                Datensatz  & \emph{R P MAE } & \emph{R U MAE} & \emph{M P MAE} & \emph{M P MAE} & 
\emph{R P RMSE} & \emph{R U RMSE} & \emph{M P RMSE} & \emph{M P RMSE} \\ \hline
\emph{auto-price}  & 2096.37 & 2075.07& 1403.20 & 1466.56 & 3336.37  & 3287.12 & 2094.59 & 2171.16 \\ \hline
\emph{concrete}    & 6.77 & 6.48& 4.27 & 4.74 &8.68 & 8.33 & 5.89 & 6.36 \\ \hline
\emph{housing}     & 3.29 & 3.20& 2.39 & 2.50 & 4.82 & 4.72 & 3.71 & 3.75 \\ \hline
\emph{stock}       & 1.19 & 1.17& 0.67 & 0.67 & 1.60 & 1.59 & 0.93 & 0.94 \\ \hline
\emph{winequality} & 0.55 & 0.53& 0.51 & 0.55 & 0.72 & 0.70 & 0.68 & 0.71   
\end{tabular}
\caption{R: regression, M: model, U: unpruned, P: pruned, MAE: mean absolute error, RMSE: root mean squared error}
\end{table}



java -cp WEKAPATH weka.classifiers.trees.M5P -R -N -M 1 -i -t regression.arff -T regression\_test.arff


MAE: 0.61 
RMSE: 0.64 

%section sec:aufgabe_6_regressionsbaume (end) 

\section{Aufgabe 7}

commandline: 

java -cp WEKAPATH weka.classifiers.Bagging -W weka.classifiers.trees.J48 -i -t contact\_lenses.arff -- -U

\section{Aufgabe 8}

numerische attribute entfernt: fnlwgt, education-num

andere attribute entfernt: marital-status, capital-gain, capital-loss, sex

sortiert nach lift: 

relationship=Not-in-family 12583 ==> class=<=50K 11307    conf:(0.9) < lift:(1.18)> lev:(0.04) [1734] conv:(2.36)



numerische attribute entfernt: fnlwgt, education-num

andere attribute entfernt: marital-status, capital-gain, capital-loss, sex, age, relationship, native-country

sortiert nach confidence:

class=>50K 11687 ==> race=White 10607    conf:(0.91)

Da 85 \% der Teilnehmer der Studie race = white hatten, wurden sehr viele Regeln der Form ? => race = white gefunden. Entfernt man dieses Attribut, findet man interessantere Regeln.


attributes \u"brig: age, education, marital-status, occupation

1. age=0 9627 ==> marital-status=Never-married 8229    conf:(0.85)

2. age=3 8296 ==> marital-status=Married-civ-spouse 5139    conf:(0.62)


attributes: marital-status, relationship, class

class=>50K 11687 ==> marital-status=Married-civ-spouse 9984    conf:(0.85)

relationship=Own-child 7581 ==> class=<=50K 7470    conf:(0.99)






\section{Aufgabe 9 | Pre-Processing}
%!TEX root = main.tex
Datensets: ionosphere, iris, yeast, letter.

Accuracy
\begin{tabular}{l|c|c|c|c}
	               & ionosphere & iris  & yeast & letter \\ \hline
J48 unfiltered     &  91.5      &  96.0 &  56.0 &  88.0  \\ \hline
J48 filtered       &  89.2      &  94.0 &  59.1 &  78.6  \\ \hline
FilteredClassifier &  91.2      &  93.3 &  57.0 &  78.7  \\ \hline
\end{tabular}\\ \\

Number of Leaves
\begin{tabular}{l|c|c|c|c}
	               & ionosphere & iris  & yeast & letter \\ \hline
J48 unfiltered     &  18        &  5    &  185  &  1226  \\ \hline
J48 filtered       &  21        &  3    &  64   &  9624  \\ \hline
FilteredClassifier &  21        &  3    &  64   &  9624  \\ \hline
\end{tabular}\\ \\



\end{document}

%!TEX root = main.tex
\documentclass
[
	a4paper,
	11pt,				% 10pt, 11pt oder 12pt - Standard ist 10pt
	pointlessnumbers,	% kein abschlie�enden Punkt hinter den Nummerierungen machen
	%pdftex,
	%chapterprefix,		% Kapitel anschreiben als Kapitel
	%openright, 		
	%openany,
	twoside,
	abstracton,
	final,				%draft,
	%normalheadings,	% �berschriften etwas kleiner (smallheadings)
	%BCOR21mm,			% Bindekorrektur, bspw. 1 cm
	%DIV13,			% f�hrt die Satzspiegelberechnung neu aus
	bibtotoc			% Literaturverzeichniss in toc eintragen
	%bibgerm,			% wegen BibTex in Deutschen Stil, sollte an alle weitergegeben werden
]
{scrreprt}				% KOMA-Script 

%\pagestyle{headings}			% sieht gut aus

% Packages:
%\usepackage[ngerman]{babel}
\usepackage[USenglish]{babel}
\usepackage[T1]{fontenc}		% f�r T1 Zeichensatz
\usepackage{lmodern} 			% Type1-Schriftart f�r nicht-englische Texte

\usepackage{amsmath}
\usepackage{amsthm}
\usepackage{amssymb}
%\usepackage{dsfont}				% brauche ich bis jetzt nur f�r die Zahlenbereiche: N,R,Z 
%\usepackage[babel,german=quotes]{csquotes}	% f�r einheitliche Anf�hrungszeichen

\usepackage{tikz}		% syntax layer for pgf, a tae macro package for generating graphics
\usetikzlibrary{plotmarks}
\usepackage{verbatim}

%\usetikzlibrary{chains,decorations}
\usepackage{pgf}
\usepackage{xcolor}				% f�r Farbmischungen
\usepackage{multicol}			% f�r mehrere Spalten
\usepackage{url}	
\usepackage{nameref}			% um Kapitel�berschriften referenzieren zu k�nnen

%\renewcaptionname{ngerman}{\abstractname}{Kurzfassung}

%\usepackage{caption}	% ich mache lieber alles �ber das Koma Script:
	\setcapindent{0em}	% kein Einzug
	\addtokomafont{caption}{\footnotesize} %\footnotesize \small 
	\addtokomafont{captionlabel}{\sffamily\bfseries}

%% Packages f�r Grafiken & Abbildungen %%%%%%%%%%%%%%%%%%%%%%
\usepackage{graphicx} %%Zum Laden von Grafiken
\usepackage{subfig} %%Teilabbildungen in einer Abbildung
\usepackage{float}		% f�r den [H] exakt hier Specifier
\usepackage{pdfpages}



% wichtig: hyperef muss als letztes Paket geladen werden, weiter Optionen: [pdfstartview={Fit}, pagebackref]
%\xdefinecolor{urlFarbe}{rgb}{0.0,0.5,0.5}
\usepackage[pdfauthor={Fabian Langguth}, bookmarks=false, pdftex, colorlinks=true, urlcolor=urlFarbe, linkcolor=black, citecolor=black]{hyperref}



\areaset[66pt]{415pt}{600pt}


\begin{document}
\title{WEKA BLAAAA}
\author{Fabian Langguth, Sebastian Koch}
\date{November 2010}

\maketitle

\section{Aufgabe 1 Regellernen} % (fold)
\label{sec:aufgabe_1_regellernen}

F\"ur diese Aufgabe benutzen wir die Datens\"atze \emph{glass, iris} und \emph{splice}.
\emph{glass} wurde f\"ur den Prism-Learner mit dem Discretize-Filter verwendet (\emph{java weka.filters.supervised.attribute.Discretize -i glass.arff -o glass\_nom.arff -R 1,2,3,4,5,6,7,8,9 -c last}).\\

Anzahl der Regeln \\
\begin{tabular}{c|c|c|c}
	             & glass & iris & splice \\ \hline
Conjunctive Rule &   1   &  1   &   1    \\ \hline
JRip             &   8   &  4   &   14   \\ \hline
Prism	         &       &      &        \\
\end{tabular}\\ \\

Gesamtanzahl der Bedingungen \\
\begin{tabular}{c|c|c|c}
	             & glass & iris & splice \\ \hline
Conjunctive Rule &   2   &  1   &   1    \\ \hline
JRip             &  18   &  3   &   55   \\ \hline
Prism	         &       &      &        \\
\end{tabular}\\ \\

Anzahl der vorhergesagten Klassen \\
\begin{tabular}{c|c|c|c}
	             & glass & iris & splice \\ \hline
Conjunctive Rule &   1   &  1   &   1    \\ \hline
JRip             &   6   &  3   &   3    \\ \hline
Prism	         &   6   &      &        \\
\end{tabular}\\ \\


Eine Default Rule existiert nur bei JRip. Dort wird als Defaultklasse \"ublicherweise die Klasse gew\"ahlt, die am h\"aufigsten im Datensatz vorkommt. 

Der Datensatz \emph{iris} l\"asst sich am einfachsten lernen, da man hier besonders wenig Regeln und besonders wenig Bedingungen ben\"otigt.


% section aufgabe_1_regellernen (end)

\section{Aufgabe 2 Evaluation von Regellernern} % (fold)
\label{sec:aufgabe_2_evaluation_von_regellernern}

\begin{tabular}{c|c|c|c|c|c}
				Datensatz         & 1x5  & 1x10 & 1x20 & LOO  & Trainingsmenge   \\ \hline
				\emph{glass}      & 67.3 & 61.8 & 60.7 & 61.7 & 85.98   \\ \hline
				\emph{iris}       & 92.0 & 88.0 & 96.0 & 93.3 & 96.0    \\ \hline
				\emph{audiology}  & 67.3 & 66.4 & 69.9 & 69.9 & 76.1    \\ \hline
				\emph{ionosphere} & 89.2 & 92.0 & 90.0 & 89.2 & 100   \\ \hline
				\emph{yeast}      & 56.5 & 57.7 & 57.7 & 59.4 & 67.8   \\ \hline
\end{tabular}

% section aufgabe_2_evaluation_von_regellernern (end)


\end{document}
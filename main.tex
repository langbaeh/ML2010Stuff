\input{header}


\areaset[66pt]{415pt}{600pt}


\begin{document}
\title{WEKA BLAAAA}
\author{Fabian Langguth, Sebastian Koch}
\date{November 2010}

\maketitle

\section{Aufgabe 1 Regellernen} % (fold)
\label{sec:aufgabe_1_regellernen}

F\"ur diese Aufgabe benutzen wir die Datens\"atze \emph{glass, iris} und \emph{splice}.
\emph{glass} wurde f\"ur den Prism-Learner mit dem Discretize-Filter verwendet (\emph{java weka.filters.supervised.attribute.Discretize -i glass.arff -o glass\_nom.arff -R 1,2,3,4,5,6,7,8,9 -c last}).\\

Anzahl der Regeln \\
\begin{tabular}{c|c|c|c}
	             & glass & iris & splice \\ \hline
Conjunctive Rule &   1   &  1   &   1    \\ \hline
JRip             &   8   &  4   &   14   \\ \hline
Prism	         &   63  &  16   &  3176    \\
\end{tabular}\\ \\

Gesamtanzahl der Bedingungen \\
\begin{tabular}{c|c|c|c}
	             & glass & iris & splice \\ \hline
Conjunctive Rule &   2   &  1   &   1    \\ \hline
JRip             &  18   &  3   &   55   \\ \hline
Prism	         &  385  & 51   &   3176   \\
\end{tabular}\\ \\

Anzahl der vorhergesagten Klassen \\
\begin{tabular}{c|c|c|c}
	             & glass & iris & splice \\ \hline
Conjunctive Rule &   1   &  1   &   1    \\ \hline
JRip             &   6   &  3   &   3    \\ \hline
Prism	         &   6   &  3   &   3    \\
\end{tabular}\\ \\


Eine Default Rule existiert nur bei JRip. Dort wird als Defaultklasse \"ublicherweise die Klasse gew\"ahlt, die am h\"aufigsten im Datensatz vorkommt. 

Der Datensatz \emph{iris} l\"asst sich am einfachsten lernen, da man hier besonders wenig Regeln und besonders wenig Bedingungen ben\"otigt.


% section aufgabe_1_regellernen (end)

\section{Aufgabe 2 Evaluation von Regellernern} % (fold)
\label{sec:aufgabe_2_evaluation_von_regellernern}

\begin{tabular}{c|c|c|c|c|c}
				Datensatz         & 1x5  & 1x10 & 1x20 & LOO  & Trainingsmenge   \\ \hline
				\emph{glass}      & 67.3 & 61.8 & 60.7 & 61.7 & 85.98   \\ \hline
				\emph{iris}       & 92.0 & 88.0 & 96.0 & 93.3 & 96.0    \\ \hline
				\emph{audiology}  & 67.3 & 66.4 & 69.9 & 69.9 & 76.1    \\ \hline
				\emph{ionosphere} & 89.2 & 92.0 & 90.0 & 89.2 & 100   \\ \hline
				\emph{yeast}      & 56.5 & 57.7 & 57.7 & 59.4 & 67.8   \\ \hline
\end{tabular}
% TODO: Qualitaet einschaetzen

\begin{tabular}{c|c|c|c|c|c}
				Datensatz         & 10x10    \\ \hline
				\emph{glass}      & 67.3     \\ \hline
				\emph{iris}       & 92.0     \\ \hline
				\emph{audiology}  & 67.3     \\ \hline
				\emph{ionosphere} & 89.2     \\ \hline
				\emph{yeast}      & 56.5     \\ \hline
\end{tabular}

% TODO: fuerht eine geschickte auswahl zu besseren abschaetzungen?

\begin{tabular}{c|c}
				Datensatz         & Validierungsmenge    \\ \hline
				\emph{glass}      & 73.1     \\ \hline
				\emph{iris}       & 93.3     \\ \hline
				\emph{audiology}  & 67.3   \\ \hline
				\emph{ionosphere} & 84.6   \\ \hline
				\emph{yeast}      & 56.1 \\ \hline
\end{tabular}

% TODO: wie war die abschaetzung?
% 10x10 Cross-Validation war grob am genausten

% section aufgabe_2_evaluation_von_regellernern (end)
\section{Aufgabe 3 ROC-Kurven}
\label{sec:aufgabe_3_ROC_kurven}

Datensatz: glass

\begin{tabular}{c|c|c|c|c}
				Regellerner       & \emph{build wind float} & \emph{containers} & \emph{tableware}  \\ \hline
				\emph{J48}			& 0.81 & 0.87 & 0.93  \\ \hline
				\emph{Naive Bayes}  & 0.71 & 0.84 & 0.98  
\end{tabular}


% section aufgabe_3_ROC_kurven (end)
\section{Aufgabe 4 Entscheidungsb\"aume}
\label{sec:aufgabe_4_entscheidungsbaume}

Datensatz: contact lenses, kr-vs-kp

Area under ROC curve

\begin{tabular}{c|c|c|c}
				Regellerner       & \emph{soft} & \emph{hard} & \emph{none}  \\ \hline
				\emph{J48 - unpruned} &  &  &   \\ \hline
				\emph{J48 - pruned}  &  &  &  \\ \hline
				\emph{ID3}  &  &  &  \\ \hline
\end{tabular}

Accuracy

\begin{tabular}{c|c|c}
				Regellerner       & \emph{contact lenses} & \emph{kr-vs-kp}  \\ \hline
				\emph{J48 - pruned} &   &   \\ \hline
				\emph{J48 - unpruned}  & &  \\ \hline
				\emph{ID3}  &  \\ \hline
\end{tabular}




%section aufgabe_4_entscheidungsbaume (end)

\section{Aufgabe 5 Nearest Neighbour}
\label{sec:aufgabe_4_nearest_neighbour}

Accuracy

\begin{tabular}{c|c|c}
                k-NN       & \emph{contact lenses} & \emph{kr-vs-kp}  \\ \hline
				\emph{k = 1} & 79.17  & 96.28  \\ \hline
				\emph{k = 3} & 79.17  & 96.50  \\ \hline
				\emph{k = 5} & 66.67  & 96.03  \\ \hline
				\emph{k = 7} & 58.33  & 95.40  \\ \hline
				\emph{k = 9} & 58.33  & 95.24  \\ \hline
				\emph{k = 11} & 58.33  & 95.06  \\ \hline
\end{tabular}

%section aufgabe_4_nearest_neighbour (end)


\section{Aufgabe 6 Regressionsb\"aume}
\label{sec:aufgabe_6_regressionsbaume}


Commandlines: 

regression tree, unpruned
java -cp WEKAPATH weka.classifiers.trees.M5P -R -N -i -t Regression/housing.arf

regression tree, pruned
java -cp WEKAPATH weka.classifiers.trees.M5P -R -i -t Regression/housing.arf

model tree, pruned
java -cp WEKAPATH weka.classifiers.trees.M5P -i -N -t Regression/housing.arf

model tree, pruned
java -cp WEKAPATH weka.classifiers.trees.M5P -i -t Regression/housing.arf

Regression tree

\begin{tabular}{c|c|c|c|c}
                Datensatz  & \emph{pruned MAE } & \emph{unpruned MAE} & \emph{pruned RMSE} & \emph{unpruned RMSE}  \\ \hline
				\emph{auto-price}  & 2096.37 & 2075.07 & 3336.37  & 3287.12 \\ \hline
				\emph{concrete}    & 6.77 & 6.48 &8.68 & 8.33  \\ \hline
				\emph{housing}     & 3.29 & 3.20 & 4.82 & 4.72 \\ \hline
				\emph{stock}       & 1.19 & 1.17 & 1.60 & 1.59 \\ \hline
				\emph{winequality} & & & & 
\end{tabular}



%section sec:aufgabe_6_regressionsbaume (end) 






\end{document}

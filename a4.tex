
Datens\"atze: glass (discretize filter), kr-vs-kp

\textbf{Fl\"ache unter ROC Kurve}
\begin{table}[htb]
	\centering
\begin{tabular}{c|c|c|c|c|c|c}
				Regellerner       & \emph{bwf} & \emph{bwn} & \emph{vwf}  & \emph{cont} & \emph{table} & \emph{head} \\ \hline
				\emph{J48 - unpruned}& 0.73 & 0.70 & 0.71 & 0.87 & 0.92 & 0.92 \\ \hline
				\emph{J48 - pruned}  & 0.77 & 0.73 & 0.73 & 0.86 & 0.87 & 0.84 \\ \hline
				\emph{ID3}           & 0.62 & 0.70 & 0.59 & 0.81 & 0.77 & 0.88 
\end{tabular}
\caption{glass}

\begin{tabular}{c|c|c}
				Regellerner       & \emph{won} & \emph{nowin} \\ \hline
				\emph{J48 - unpruned} & 1.0 & 1.0  \\ \hline
				\emph{J48 - pruned}  & 1.0 & 1.0  \\ \hline
				\emph{ID3}  & 1.0 & 1.0 
\end{tabular}
\caption{kr-vs-kp}
\end{table}

\textbf{Accuracy}
\begin{table}[htb]
	\centering
\begin{tabular}{c|c|c}
				Regellerner       & \emph{glass} & \emph{kr-vs-kp}  \\ \hline
				\emph{J48 - unpruned}  & 57.94 & 99.41 \\ \hline
				\emph{J48 - pruned} & 57.94  & 99.44 \\ \hline
				\emph{ID3}  & 50.47 & 99.69
\end{tabular}
\end{table}

\textbf{Gr\"osse der entstandenen B\"aume}

\begin{table}[htb]
	\centering
\begin{tabular}{c|c|c}
	Regellerner       & \emph{size of the tree} & \emph{number of leaves}  \\ \hline
	\emph{J48 - unpruned} & 221  & 199  \\ \hline
	\emph{J48 - pruned}   & 81   & 73   \\ \hline
	\emph{ID3}            & 550  & 496  
\end{tabular}
\caption{glass}

\begin{tabular}{c|c|c}
	Regellerner       & \emph{size of the tree} & \emph{number of leaves}  \\ \hline
	\emph{J48 - unpruned}  & 82 & 43 \\ \hline
	\emph{J48 - pruned} & 59  & 31 \\ \hline
	\emph{ID3}  & 95 & 49
\end{tabular}
\caption{kr-vs-kp}
\end{table}


Betrachtet man die Fl\"ache unter der ROC Kurve so hat \emph{ID3} im Allgemeinen schlechtere Werte. Dies spiegelt sich auch in der Accuracy wieder. Au\ss erdem erzeugt \emph{ID3} immer einen gr\"o\ss eren Baum. Man kann also deutlich erkennen, das ein gro\ss er Baum schlecht veralgemeinert und somit auch schlechtere Performancewerte erzielt. Das Pruning von \emph{J48} erzeugt einen kleineren Baum, der allerdings nicht deutlich bessere Performance liefert. Dieser Effekt ist dabei wahrscheinlich stark abh\"angig von den von uns gew\"ahlten Datens\"atze.